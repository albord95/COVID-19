\documentclass{article}
\usepackage[utf8]{inputenc}
\usepackage[a4paper]{geometry}                % See geometry.pdf to learn the layout options. There are lots.
\geometry{letterpaper}                   % ... or a4paper or a5paper or ... 
\usepackage{graphicx}
\usepackage{amssymb}
\usepackage{epstopdf}
\usepackage{amsmath} 
\usepackage{latexsym}
\usepackage{amsthm} 
\usepackage{eucal} 
%\usepackage{eufrak} 
\usepackage{float}	
\usepackage{subfig}	
\usepackage{physics}
\usepackage{wrapfig}
 \usepackage{amsbsy}
 %\usepackage{mathrsfs}
 \usepackage{bbold}
\usepackage{floatflt,epsfig}
%\usepackage[latin1]{inputenc}
\usepackage{caption}
\usepackage{braket}
\usepackage{quoting} %citazioni
\quotingsetup{font=small}
\usepackage[english]{babel}
\usepackage{cancel}
\theoremstyle{definition}
%\newtheorem{definizione}{Definizione}[chapter]
\theoremstyle{plain}
%\newtheorem{teorema}{Teorema}[chapter]
\theoremstyle{plain}
%\newtheorem{proposizione}{Proposizione}[chapter]

\usepackage{cite}

\usepackage{amscd}
\usepackage{slashed}
\usepackage{mathtools}
\usepackage{centernot}

\usepackage{verbatim}


\usepackage[pdfa]{hyperref}
\hypersetup{
colorlinks,
citecolor=black,
filecolor=black,
linkcolor=black,
urlcolor=black}
%\usepackage[english]{varioref}
\usepackage{newlfont}
\usepackage{color}


\providecommand{\abstract}{} 
 %\providecommand{\abstractname}{Abstract} 
\usepackage{abstract}

\newcommand{\dang}{\mathfrak{d}}
\newcommand{\exv}[1]{\mathbb{E}[#1]}

\title{The MILO model}
\author{Cristoforo Iossa \\ Andrea Marino \\ Piero Lafiosca \\ Alessandro Marco Oliva }
\date{}


\begin{document}


\maketitle
\begin{center}
\today
\end{center}

\begin{abstract}
    Our aim is trying to build a model that describes the COVID-19 epidemic and makes possible to forecast some data in the upcoming days.
\end{abstract}

\section{The main ideas}
The main idea is that la mamma di piero è puttana

\section{A Hubei model}
Here we take into account saturation effects combined with lockdown measures. The region is considered as a unique area.

\subsection{Variables}

\begin{itemize}
\item $h= 5.85 \cdot 10^7$ is the number of inhabitants;
\item $r= 18$ or $19$ is the average time of recovery;
\item $t= 5$ or $6$ is the average time of incubation;
\item $E(n)$ is the number of "exposed" (incubati) at time $n$;
\item $I(n)$ is the number of infected at time $n$;
\item $R(n)$ is the number of recovered at time $n$;
\item $s$ is the diffusion parameter (the growth rate of the exponential at the beginning);
\item $\alpha$ is the lockdown parameter;
\item $S(n)$ is the auxiliary variable of "susceptibles", i.e. everyone minus exposed, infected, recovered. In future the derivative of $S(n)$ could be negative due to deaths.

\end{itemize}

\subsection{Formula}
The model for the evolution is then given by

\begin{align}
    E(n+1)  = & E(n)-I(n) + \left [ 1-\left (1-\frac{s\alpha}{h} \right )^{I(n) / \alpha}  \right ]( h- E(n) - I(n) - R(n) ] \\
    I(n+1) = & E(n+1-t)- R(n) \\
    R(n+1) = & I(n+1-r) 
\end{align}

Beware: data must be MULTIPLIED BY CINQUE PRIMA DE doing the fit, to take into account the discrepancy between (positive tests) $\<--\> $ (actual infected). The phenomenon is NOT LINEAR PORCA TROIA QUINDI NUN commute.

Also, maybe the parameter $s$ could be fitted oon the initial days of diffusion, where the evolution is exponential. In this case formula (1) simplifies to

$$     E(n+1)  = (1+s) E(n)-I(n)   $$

\section{Correction of the Formula}
Here we settle a rough continous model for transition among states to correct the formula.

\begin{itemize}
    \item $\delta_{SE}(t)$: susceptibles that turns exposed in the interval $t, t+dt$.
    \item $\delta_{EI}(t)$: exposed that turns infected in the interval $t, t+dt$.
    \item $\delta_{IR}(t) $: infected turn recovered in teh interval bla.
    \item $ \delta I (t) = I(t+dt) - I(t)$ and similarly;
\end{itemize}

Now we note that
\begin{align}
    \delta S(t) & = - \delta_{SE}(t) \\
    \delta E(t) & = - \delta_{EI}(t) + \delta_{SE}(t) \\
    \delta I (t) & = - \delta_{IR}(t) + \delta_{EI}(t) \\
    \delta R(t) & = \delta_{IR}(t) 
\end{align}
Also, we have that:
\begin{align}
    \int_{n}^{n+1} \delta_{SE}(t) & \simeq \left ( 1- \frac{s\alpha}{h} \right )^{I(n) / \alpha} S(n)  \\
    \delta_{EI}(t) & = \delta_{SE}(t- t_0) \\
    \delta_{IR}(t) & = \delta_{EI}(t-r_0)
\end{align}
Set now
\begin{align}
\Delta_{SE}(n) & = \int_{n}^{n+1}  \delta_{SE}(t) dt = \left ( 1- \frac{s\alpha}{h} \right )^{I(n) / \alpha} S(n) \\
   \Delta_{EI}(n) & = \int_{n}^{n+1}  \delta_{EI}(t) dt  = \int_{n}^{n+1} \delta_{SE}(t-t_0) dt = \Delta_{SE}(n-t_0)\\
   \Delta_{IR}(n) & = \int_{n}^{n+1} \delta_{IR} (t) dt = \int_{n}^{n+1} \delta_{EI} (t-r_0) dt = \Delta_{EI}(n-r_0) 
\end{align}
Integrating between $n$ and $n+1$ equalities 4-7 we get
\begin{align}
    S(n+1) - S(n) & = - \Delta_{SE}(n) \\
    E(n+1) - E(n) & = - \Delta_{EI}(n) + \Delta_{SE}(n) \\
    I(n+1)-I(n) & = - \Delta_{IR}(n) + \Delta_{EI}(n) \\
    R(n+1)-R(n) & = \Delta_{IR}(n) 
\end{align}

The program can be implemented by firstly calculating $\Delta$'s with equations 11-13, then computing 14-17. It is pleasantly manifest from these equations that the number of inhabitants (S+E+I+R) is costant, cause variations simplify.

\section{Corrections on lockdown}

We analyze here the hidden dynamics of infection in the parameter $s$. Set
$$ \delta s = \text{ number of wannabe infected from a single person in the interval } t, t+dt $$

Suppose there is a number $N_{\dang}$ of dangerous situations one can go through. Every situation $x$ has a danger $\dang(x)$ that the infection vada a buon fine. Each situation occurs with probability $p(x)$ in an average day of an average citizen of an average LA MAMMA DI PIERO. We set
$$ \dang = N_{\dang} \exv{\dang} = \sum_{x\  SITUA} \dang(x) p(x)  $$
as the \textit{cumulated risk} of infection in a day.
Finally, we set $\rho$ the infectivity of coronavirus, which encodes the probability that a dangerous situation results in an effective infection. We have that 

$$ \delta s = \sum_{x \in \text{SITUE} } \rho \dang(x) p(x) = \rho \dang  $$

When lockdown switch on, the probabilites go down. This results in a new $\dang_{ld} < \dang$. We set $$\beta = \dang / \dang_{ld} > 1 $$
as the \textit{lockdown safety factor}. We then have that

$$ \delta s_{ld} = \rho \dang_{ld} = \frac{\delta s}{\beta} $$

Now we want to better understand how $\beta$ varies wrt $\alpha$, which is the number of virtual meeting points. The following is a rough model that can give an idea. \newline

Suppose our city has $N$ meeting points. Each person normally attend $k$ of these meeting points for an average amount of time T. When the city is locked down, there are two main consequences:
\begin{itemize}
    \item $k$ decreases: we don't go to work, we don't go to CUT THE FUCKING HAIRS.. This results in a $k'$
    \item $T$ decreases: most of the time we are at home;
    \item $N$ becomes $N/\alpha$, where $\alpha$ is the number of virtual cities; we can no more go to a supermarket dall'altra parte della città.
\end{itemize}

Let's calculate the probability of getting in 

E cazzo fra questo giustifica l'alpha sotto. Comunque è uguale metterlo all'esponente o dentro basta che sta al denominatore. Infatti siamo nell'approssimazione logistica perchè il parametro è piccolino e donc

$$ \left ( 1- \frac{s}{h} \right )^{1/\alpha} \simeq  \left (1- \frac{s}{h \alpha} \right ) $$

Questo ci dice pure che possiamo pisciare questo fattore buffo e seccare tutto al logistico. Ahimè. E' così. Ma intanto è la VERITA

Formmule aggiornate:

\textbf{Variations}

\begin{align}
\Delta_{SE}(n) & =  \left ( 1- \frac{s I(n)}{h \alpha}  \right ) S(n) \\
   \Delta_{EI}(n) & =  \Delta_{SE}(n-t_0)\\
   \Delta_{IR}(n) & =  \Delta_{EI}(n-r_0) 
\end{align}

\textbf{States}

\begin{align}
    S(n+1) - S(n) & = - \Delta_{SE}(n) \\
    E(n+1) - E(n) & = - \Delta_{EI}(n) + \Delta_{SE}(n) \\
    I(n+1)-I(n) & = - \Delta_{IR}(n) + \Delta_{EI}(n) \\
    R(n+1)-R(n) & = \Delta_{IR}(n) 
\end{align}

\end{document}
